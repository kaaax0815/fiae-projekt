% !TEX root = ../Projektdokumentation.tex
\section{Einleitung}
\label{sec:Einleitung}


In der folgenden Projektdokumentation wird der Ablauf des Abschlussprojektes, das durch den Autor
im Rahmen seiner Abschlussprüfung zum Fachinformatiker mit der Fachrichtung Anwendungsentwicklung durchgeführt wurde, erläutert.


\subsection{Projektumfeld} 
\label{sec:Projektumfeld}
	Der Ausbildungsbetrieb \OFF ist ein zukunftsorientiertes Unternehmen mit Spezialisierung auf Marketing und digitale Transformation.
	Im Mittelpunkt der Geschäftstätigkeit steht die strategische und technische Beratung von Unternehmen,
	die ihre Geschäftsprozesse modernisieren, optimieren und digitalisieren möchten.
	\OFF versteht sich als Schnittstelle zwischen Technologie und Unternehmensentwicklung und begleitet seine Kunden
	auf dem Weg in die digitale Zukunft – von der ersten Analyse bis zur erfolgreichen Implementierung maßgeschneiderter Lösungen.
	
	\OFF plant, \SCO zu vertreiben – ein System zur Einrichtung, Verwaltung und Nutzung von \ac{LoRaWAN} Netzwerken.
	Mit \SCO können Sensordaten effizient erfasst, verarbeitet, analysiert und visualisiert werden.
	Das System bietet eine ganzheitliche Lösung zur Digitalisierung und Überwachung von Prozessen in verschiedenen Anwendungsbereichen.


\subsection{Projektziel} 
\label{sec:Projektziel}
	Das Projektziel bestand darin ein ganzheitliches System zur Erfassung, Verwaltung und Visualisierung von Sensordaten zu erstellen.
	Das System ermöglicht die zentrale Integration und Administration von Gateways und Sensoren über ein webbasiertes Management-Portal.
	Erfasste Daten werden über ein Backend verarbeitet, gespeichert und für die Visualisierung in einer intuitiven Web- und Mobile-App aufbereitet.
	
	Folgende Aspekte lagen im Fokus:
	\begin{itemize}
		\item zentrale Verwaltung von Gateways und Sensoren
		\item zuverlässige Datenübertragung und -verarbeitung mithilfe des \ac{LoRaWAN} Protokolls
		\item Backend zur Datenaufnahme, Analyse und Bereitstellung von Schnittstellen
		\item benutzerfreundliche Visualisierung der Sensordaten durch eine App für Android und iOS
	\end{itemize}
	
	Das Projekt zielte darauf ab die bestehenden Systeme (\dahe Backend, Web und App) um eine modulare,
	erweiterbare und benutzerorientierte \ac{IoT}-Plattform zu erweitern,
	die in unterschiedlichen Anwendungsfeldern wie Smart City, Umweltmonitoring oder Industrie 4.0 flexibel eingesetzt werden kann.


\subsection{Projektbegründung} 
\label{sec:Projektbegruendung}
	Im Alltag vieler Unternehmen gehört das manuelle Ablesen von Strom-, Wasser- oder Verbrauchszählern noch immer zum Standard.
	Auch im Umfeld der offizium next GmbH gibt es Kunden, bei denen beispielsweise Stromzähler unterschiedlicher Stromkreisläufe
	regelmäßig von Mitarbeitenden vor Ort händisch abgelesen und dokumentiert werden müssen.
	Diese Vorgänge sind nicht nur zeitintensiv, sondern auch fehleranfällig,
	da sie auf manuelle Übertragung und Interpretation von Messwerten angewiesen sind.

	Mit dem System \SCO sollte genau hier angesetzt werden:
	Statt wiederkehrende Routinetätigkeiten durch Personal durchführen zu lassen,
	können die benötigten Verbrauchsdaten automatisiert erfasst, übertragen, gespeichert und aufbereitet werden.
	Das reduziert nicht nur den Arbeitsaufwand für Mitarbeitende erheblich,
	sondern sorgt gleichzeitig für eine höhere Datenqualität und eine bessere Nachvollziehbarkeit.
	Falsche Ablesewerte, vergessene Protokolle oder uneinheitliche Formate werden somit weitestgehend vermieden.


\subsection{Projektschnittstellen} 
\label{sec:Projektschnittstellen}
	Für die Übertragung, Verarbeitung und Darstellung der Sensordaten innerhalb des Systems waren mehrere technische Schnittstellen erforderlich,
	die unterschiedliche Protokolle und Dienste miteinander verbinden.
	
	Die von den Sensoren erfassten Daten werden über LoRaWAN-Gateways an ChirpStack, eine Open-Source-Netzwerkserver-Plattform für LoRaWAN, übermittelt. 
	Um diese Daten weiterzuverarbeiten, wurde ein Webhook eingerichtet, der die dekodierten Daten automatisch an das Backend-System weiterleitet. 
	Dort erfolgt die Validierung, Speicherung und Aufbereitung der empfangenen Informationen.
	
	Da das entwickelte Backend nur \ac{HTTP} unterstützt, ChirpStack jedoch ausschließlich über eine gRPC-Schnittstelle verwaltet werden kann,
	war die Implementierung einer Vermittlungsschicht erforderlich.
	Diese Komponente übernimmt die Aufgabe, eingehende Anfragen des Backends in entsprechende gRPC-Aufrufe umzuwandeln.
	So wird eine zentrale und einfache Verwaltung der LoRaWAN-Komponenten direkt über das Backend ermöglicht.
	
	Zusätzlich wurde eine \ac{REST}-\ac{API} entwickelt, über die sowohl die Webanwendung als auch die mobile App
	auf die verarbeiteten Sensordaten zugreifen können. Diese Schnittstelle stellt die Daten in strukturierter Form bereit
	und bildet die Grundlage für die nutzerfreundliche Visualisierung innerhalb der Benutzeroberflächen.


\subsection{Projektabgrenzung} 
\label{sec:Projektabgrenzung}
	Da der Projektumfang begrenzt ist, ist die Erstellung der Weboberfläche zur einfachen Verwaltung kein Bestandteil des Projekts.
