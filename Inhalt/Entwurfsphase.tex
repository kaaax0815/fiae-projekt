% !TEX root = ../Projektdokumentation.tex
\section{Entwurfsphase} 
\label{sec:Entwurfsphase}

\subsection{Zielplattform}
\label{sec:Zielplattform}
	Wie bereits unter \Verweis{sec:Projektziel} erwähnt, sollen bestehende Systeme erweitert werden.
	Dies macht die Entwicklung einfacher, da bestehende Funktionalitäten aufgegriffen werden können.
	Daten aus dem System liegen in einer MySQL Datenbank. Auf ein neues Datenbanksystem kann daher verzichtet werden.

	Als Programmiersprache im Backend wird \ac{PHP} verwendet, da das bestehende System dies vorgibt.
	Die App verwendet NativeScript-Vue somit wird die App in JavaScript und Vue.js programmiert.


\subsection{Architekturdesign}
\label{sec:Architekturdesign}
	Im Backend der Anwendung kommt das \ac{PHP}-Framework Laravel zum Einsatz, welches auf dem \ac{MVC}-Architekturmuster basiert.
	Diese bewährte Architektur wird durch eine zusätzliche Komponentenstruktur erweitert, um eine modulare und wartbare Codebasis zu ermöglichen.

	Die Anwendung gliedert sich grundsätzlich in drei zentrale Schichten:

	Die Model-Schicht bildet sowohl die zugrunde liegenden Datenstrukturen als auch die fachliche Logik der Anwendung ab.
	Dabei wird Eloquent verwendet, ein Active Record \ac{ORM} von Laravel. Das bedeutet, dass die Datenbankoperationen
	direkt im jeweiligen Model definiert und ausgeführt werden, anstatt ein separates Repository-Pattern zu verwenden.
	So sind Funktionen wie das Abrufen, Speichern oder Löschen von Datensätzen eng mit den jeweiligen Datenmodellen verknüpft.

	Die View-Schicht ist in einer klassischen Laravel-Anwendung für die Darstellung der Benutzeroberfläche verantwortlich.
	In diesem Projekt wird jedoch eine moderne \ac{SPA} \bzw App auf Basis von Vue.js eingesetzt,
	wodurch die klassische Laravel-View nur noch eine untergeordnete Rolle spielt.
	Sie wird primär für die Darstellung von E-Mail-Templates oder anderen serverseitig generierten Inhalten verwendet.

	Die Controller bilden die Vermittlung zwischen Benutzeranfragen und der Geschäftslogik.
	Sie nehmen eingehende \ac{HTTP}-Anfragen entgegen, verarbeiten diese, interagieren mit den entsprechenden Models
	und geben eine passende Antwort zurück. Darüber hinaus sind die Controller auch für die Validierung von Benutzereingaben zuständig, 
	sowohl hinsichtlich formaler Korrektheit als auch, falls erforderlich, in Bezug auf deren inhaltliche Sinnhaftigkeit.

	Durch diese klar strukturierte Trennung der Verantwortlichkeiten wird eine saubere, wartbare und erweiterbare Codebasis sichergestellt,
	die sowohl die Entwicklung neuer Features als auch die Fehlerbehebung erleichtert.


	Ein weiteres zentrales Element der Systemarchitektur ist der gRPC-Proxy, welcher als Vermittler zwischen dem \ac{REST}-basierten Backend
	und dem gRPC-basierten ChirpStack-Server fungiert. Seine Hauptaufgabe besteht darin, eingehende \ac{REST}-Anfragen aus dem Backend
	entgegenzunehmen und diese in äquivalente gRPC-Aufrufe zu übersetzen. Dabei handelt es sich um eine rein technische Übersetzungsschicht
	ohne eigene Geschäftslogik. Aus diesem Grund wurde bewusst auf die Implementierung eines komplexen Software-Patterns verzichtet,
	um die Komponente so einfach und wartungsarm wie möglich zu halten.

	Für die Umsetzung des Proxys wird das Webframework h3 eingesetzt. h3 zeichnet sich durch seine hohe Performance, geringe Größe
	sowie die hervorragende Unterstützung von TypeScript aus. Der Einsatz von TypeScript ermöglicht es, viele potenzielle
	Fehler bereits beim Kompilieren zu identifizieren und damit die Stabilität und Wartbarkeit des Codes deutlich zu erhöhen.
	Zur Validierung der eingehenden \ac{REST}-Anfragen wird die Bibliothek \Ausgabe{zod} verwendet. Dadurch kann sichergestellt werden,
	dass fehlerhafte oder unvollständige Daten nicht an ChirpStack weitergeleitet werden.
	Stattdessen erhält das Backend unmittelbar eine aussagekräftige Fehlermeldung,
	was die Nutzerfreundlichkeit und Robustheit der Schnittstelle erhöht.

	
	Die letzte serverseitige Komponente ist ChirpStack, ein Open-Source Netzwerk- und Applikationsserver für LoRaWAN.
	ChirpStack ermöglicht die Verwaltung von Sensoren und Gateways sowie die Organisation und Steuerung von LoRaWAN-Sitzungen.
	Zusätzlich bietet die Plattform die Möglichkeit, benutzerdefinierte Decoder-Skripte zu hinterlegen. Diese werden genutzt,
	um die von den Sensoren gesendeten Binärdaten automatisiert in strukturierte JSON-Objekte umzuwandeln.
	Somit stellt ChirpStack eine zentrale und flexible Plattform für den zuverlässigen Betrieb
	und die Integration von LoRaWAN-basierten IoT-Systemen dar.


	Die App verwendet NativeScript-Vue, ein Framework, das Webentwicklern ermöglicht, native mobile Apps mit vertrauten Webtechnologien
	wie JavaScript zu erstellen. Ein großer Vorteil von NativeScript ist, dass der Code einmal geschrieben und anschließend
	sowohl für iOS als auch Android verwendet werden kann.

	NativeScript-Vue kombiniert NativeScript mit dem JavaScript Webframework Vue.js. Vue.js basiert auf einem reaktiven Datenmodell
	und dem Virtual DOM. Reaktivität bedeutet, dass sich die Benutzeroberfläche automatisch aktualisiert,
	sobald sich die zugrunde liegenden Daten ändern. Der Virtual DOM optimiert diesen Prozess, indem er eine virtuelle Kopie
	der Benutzeroberfläche erstellt und nur die tatsächlich veränderten Teile neu rendert. Dadurch werden Updates effizient und
	ressourcenschonend durchgeführt.

	Was NativeScript-Vue besonders macht, ist, dass es keine HTML-Webansicht auf dem Gerät anzeigt.
	Stattdessen werden alle UI-Elemente tatsächlich nativ gerendert, was zu einer deutlich besseren Performance
	und einem authentischen Look-and-Feel auf der jeweiligen Plattform führt.


\subsection{Entwurf der Benutzeroberfläche}
\label{sec:Benutzeroberflaeche}

	Da das Nutzererlebnis bei mobilen Anwendungen von zentraler Bedeutung ist,
	wurde frühzeitig ein besonderer Fokus auf die Gestaltung der Benutzeroberfläche gelegt.
	Zu diesem Zweck wurden zunächst Mockups entwickelt, die anschließend in enger Abstimmung mit dem Kunden mehrfach überarbeitet
	und verbessert wurden. Die finalen Entwürfe sind im \Anhang{app:Mockups} dokumentiert.

	Da die Anwendung in eine bestehende App integriert wird, deren grundlegendes Layout bereits definiert ist,
	mussten diese bei der Gestaltung der Benutzeroberfläche berücksichtigt werden.
	Insbesondere sind der obere Seitenbereich (Header) sowie das Navigationsmenü fest vorgegeben.
	Diese Strukturen bilden den äußeren Rahmen des App-Designs und wurden im Entwurf entsprechend berücksichtigt.

	Die Startseite der Anwendung bietet eine Übersicht über alle vorhandenen Sensoren.
	Jeder Sensor wird in einer eigenen Kachel dargestellt.
	Gemäß der mit dem Kunden abgestimmten Gestaltung ist die Kachel weiß hinterlegt und hebt sich damit optisch von dem grauen Hintergrund ab.
	Jede dieser Kacheln enthält folgende Informationen:

	\begin{itemize}
		\item den Namen des Sensors
		\item die Art des Sensors visualisiert durch ein passendes Symbol
		\item den aktuellen Messwert
		\item den Batteriestand dargestellt durch ein Symbol
		\item den Zeitpunkt der letzten Datenübertragung
	\end{itemize}

	Wählt der Nutzer einen Sensor aus, wird eine Detailansicht geöffnet.
	Dort werden die oben genannten Informationen erneut angezeigt, wobei der Batteriestand deutlicher hervorgehoben wird.
	Falls der Sensor zusätzliche Messwerte liefert, die nicht unmittelbar relevant, aber dennoch informativ sind,
	werden diese in dieser Detailansicht ebenfalls dargestellt. Darüber hinaus können Nutzer einzelne Sensoren als Favoriten markieren.
	Diese Favoriten erscheinen auf der Startseite ganz oben, um einen schnellen Zugriff zu ermöglichen.

	Ein zentrales Element der Benutzeroberfläche ist die Darstellung historischer Sensordaten.
	Diese erfolgt mithilfe eines Säulendiagramms, das in drei zeitliche Ansichten unterteilt ist: Tag, Woche und Jahr.

	\begin{itemize}
		\item In der Tagesansicht werden die Daten in zwei-Stunden-Intervallen gruppiert.
		\item In der Wochenansicht erfolgt die Aggregation der Daten nach Wochentagen.
		\item In der Jahresansicht wird der Verlauf über die Monate hinweg dargestellt.
	\end{itemize}

	Zur besseren Vergleichbarkeit wird in jeder dieser Ansichten zusätzlich der jeweilige Vorzeitraum eingeblendet.
	Dadurch können Veränderungen im zeitlichen Verlauf unmittelbar erkannt und eingeordnet werden.

	Insgesamt verfolgt der Entwurf der Benutzeroberfläche das Ziel, eine klare,
	übersichtliche und nutzerfreundliche Darstellung aller relevanten Informationen sicherzustellen.

\subsection{Datenmodell}
\label{sec:Datenmodell}
	Das Datenmodell der Anwendung umfasst die zentralen Entitäten \Klasse{Gateway}, \Klasse{Gerät}, \Klasse{Sensor}, \Klasse{Kategorie}, \Klasse{Wert},
	\Klasse{Geräte-Profil}, \Klasse{Hersteller} sowie \Klasse{Sensortyp}. Diese bilden die Grundlage für die strukturierte Erfassung,
	Zuordnung und Auswertung der erfassten Sensordaten.

	Ein \Klasse{Gerät} stellt einen physischen Sensor bzw. ein Sensormodul dar,
	das über eine eindeutige \Attribut{devEUI} identifiziert wird.
	Es fungiert als logische Einheit, die mit einem oder mehreren \Klasse{Sensoren} ausgestattet ist
	und regelmäßig Messwerte an das System übermittelt. Jedes \Klasse{Gerät} ist einem \Klasse{Geräte-Profil} zugeordnet,
	das allgemeine Informationen wie unterstützte Sensortypen oder Kommunikationsverhalten enthält.

	Das \Klasse{Geräte-Profil} wiederum ist einem bestimmten \Klasse{Hersteller} zugeordnet,
	welcher das Gerät entwickelt oder produziert hat. Durch diese Beziehung können gerätespezifische Merkmale,
	Abhängigkeiten und Kompatibilitäten nachvollziehbar dokumentiert und verwaltet werden.
	Ein \Klasse{Geräte-Profil} definiert eine Menge an unterstützten \Klasse{Sensortypen}.
	Diese Typen geben an, welche Arten von Sensoren in einem Gerät dieses Profils verbaut sind.
	Beim Anlegen eines neuen Geräts im System werden automatisch die zugehörigen \Klasse{Sensoren} entsprechend dem Geräte-Profil erstellt.

	Jeder \Klasse{Sensor} steht für eine konkrete Messeinheit innerhalb eines Geräts.
	Ein \Klasse{Gerät} kann mehrere Sensoren enthalten, da in modernen Umweltsensoren meist mehrere Messgrößen parallel erfasst werden.
	Typische Beispiele sind Temperatur, Luftfeuchtigkeit, Helligkeit und CO\textsubscript{2}-Konzentration.
	Ein \Klasse{Sensor} kann mehreren \Klasse{Kategorien} zugeordnet sein, welche thematische oder funktionale Gruppierungen darstellen.
	Kategorien dienen unter anderem der Strukturierung und Aggregation von Sensordaten.
	Ein \Klasse{Sensor} speichert kontinuierlich \Klasse{Werte}, die jeweils einen gemessenen Messwert sowie den zugehörigen Zeitstempel enthalten.
	Zusätzlich werden \Klasse{Durchschnittswerte} berechnet und gespeichert
	um aggregierte Informationen über bestimmte Zeitintervalle hinweg (\zB stündlich, täglich) effizient verfügbar zu machen.
	Diese Durchschnittswerte ermöglichen eine performante Visualisierung historischer Verläufe ohne aufwendig alle Einzelwerte laden zu müssen.
	
	Diese Beziehungen sind im \Anhang{app:ERM} als \ac{ERM} dargestellt.


\subsection{Geschäftslogik}
\label{sec:Geschaeftslogik}
	Vor Beginn der Implementierung wurde ein Komponentendiagramm erstellt, um das Zusammenspiel der verschiedenen Systembestandteile
	übersichtlich darzustellen und die zugrunde liegende Architektur zu visualisieren.

	Der Einstiegspunkt für den Nutzer ist die mobile App. Diese kommuniziert ausschließlich mit dem Laravel-Backend,
	welches als zentrale Steuerungseinheit fungiert. Über das Backend können unter anderem Sensordaten sowie deren Historie abgerufen werden.

	Darüber hinaus dient das Laravel-Backend als Schnittstelle für interne und externe Benutzer,
	beispielsweise für Mitarbeitende, die über eine definierte API, mithilfe von Postman, Aktionen wie das Anlegen neuer Gateways durchführen können.

	Müssen im Zuge solcher Aktionen auch Änderungen im ChirpStack vorgenommen werden, \zB das parallele Anlegen eines Geräts oder Gateways,
	wird die entsprechende Anfrage intern an den gRPC-Proxy weitergeleitet. Dieser übernimmt die Konvertierung der REST-Anfragen
	in gRPC-Aufrufe und leitet diese an ChirpStack weiter, welches die entsprechenden Daten ebenfalls in seinem System speichert.
	
	Eine visuelle Darstellung dieser Systemkomponenten und ihrer Beziehungen ist im \Anhang{app:KomponentenSystem} zu finden.


\subsection{Maßnahmen zur Qualitätssicherung}
\label{sec:Qualitaetssicherung}
	Im Rahmen dieses Projekts wurde bewusst auf automatisierte Tests oder den Einsatz spezieller Qualitätssicherungs-Tools verzichtet.
	Stattdessen erfolgte die Qualitätssicherung in Form manueller Tests sowie durch regelmäßige Code-Reviews während der Entwicklungsphase.

	Die manuelle Überprüfung ermöglichte eine praxisnahe Validierung der Funktionalitäten,
	insbesondere im Hinblick auf die Benutzeroberfläche und die Interaktion mit dem Backend.
	Zusätzlich trugen die Code-Reviews dazu bei, mögliche Fehler frühzeitig zu erkennen,
	die Codequalität zu verbessern und die Einhaltung von Projektstandards sicherzustellen.


\subsection{Deployment}
\label{sec:Deployment}
	Für den Betrieb von ChirpStack sowie dem gRPC-Proxy wurde ein eigener Server benötigt,
	da beide Dienste unabhängig von der Hauptanwendung laufen und bestimmte Netzwerk- sowie Systemressourcen erfordern.
	Aus diesem Grund wurde ein Cloud-Server bei Ionos angemietet.
	Da sowohl ChirpStack als auch der gRPC-Proxy vergleichsweise geringe Systemanforderungen aufweisen,
	genügte eine Konfiguration mit einer einzelnen CPU und 1 GB RAM, um eine stabile Ausführung sicherzustellen.

	Als Betriebssystem kam Debian 12 zum Einsatz.
	Diese Version stellt die derzeit aktuelle stabile Veröffentlichung der Debian-Distribution dar und wurde aufgrund ihrer Zuverlässigkeit,
	Sicherheit sowie der guten Dokumentation gewählt.\footnote{Vgl. \citet{Debian}}

	Nach der Grundkonfiguration des Servers erfolgte die Installation von ChirpStack gemäß der offiziellen Dokumentation.
	Dabei wurden auch alle erforderlichen Abhängigkeiten eingerichtet, darunter der Message-Broker Mosquitto,
	die In-Memory-Datenbank Redis sowie das relationale Datenbanksystem PostgreSQL.
	
	Für den Betrieb von dem gRPC-Proxy wurde zudem Node.js installiert.
	Zur Prozessverwaltung und zur Sicherstellung der dauerhaften Verfügbarkeit des Proxys wird \Ausgabe{pm2} verwendet.
	Dieses Tool ermöglicht es, den Dienst automatisch beim Systemstart auszuführen und bei unerwarteten Fehlern
	oder Abstürzen selbstständig neu zu starten.

	Um sowohl den Zugriff auf ChirpStack als auch auf den gRPC-Proxy über das Internet zu ermöglichen,
	wurde ein Reverse Proxy mit Nginx eingerichtet. In Kombination mit Certbot konnten zudem automatisierte SSL-Zertifikate
	von Let's Encrypt eingerichtet werden, wodurch eine verschlüsselte und sichere Kommunikation über HTTPS gewährleistet ist.

	Diese Infrastruktur gewährleistet einen zuverlässigen und wartungsarmen Betrieb der zentralen Backend-Komponenten des Projektes.


\subsection{Pflichtenheft}
\label{sec:Pflichtenheft}
	Basierend auf den erarbeiteten Entwürfen wurde zum Abschluss der Entwurfsphase ein Pflichtenheft erstellt.
	Dieses baut auf dem zuvor definierten Lastenheft auf und konkretisiert die Umsetzung der in Abschnitt~\Verweis{sec:Lastenheft} festgelegten Anforderungen.
	Es dient sowohl während der Entwicklungsphase als Orientierungshilfe als auch am Projektende zur Überprüfung, ob sämtliche Anforderungen
	erfüllt wurden. Ein Auszug des Pflichtenhefts ist im \Anhang{app:Pflichtenheft} dargestellt.
