% !TEX root = ../Projektdokumentation.tex
\section{Implementierungsphase} 
\label{sec:Implementierungsphase}

\subsection{Implementierung der Datenstrukturen}
\label{sec:ImplementierungDatenstrukturen}
	Die Umsetzung des Datenbankschemas orientierte sich weitgehend an dem in Abschnitt~\ref{sec:Datenmodell} beschriebenen \ac{ERM}.
	Lediglich kleinere Anpassungen waren notwendig, um spezifische technische Anforderungen und Rahmenbedingungen zu erfüllen.

	Zur besseren Strukturierung und Übersichtlichkeit innerhalb der Datenbank wurden die Tabellen eines Moduls mit einem gemeinsamen Präfix versehen.
	Im vorliegenden Fall wurde das Präfix \Ausgabe{LORA} verwendet, wodurch eine eindeutige Zuordnung der Tabellen zum jeweiligen Modul gewährleistet ist.

	Darüber hinaus wurden zusätzliche Spalten vorgesehen, beispielsweise zur Speicherung von Informationen über Dezimalstellen oder
	andere potenzielle Erweiterungen, um künftige Anforderungen flexibel abbilden zu können.

	Die Erstellung der Datenbankstruktur erfolgte mithilfe sogenannter Migrationen,
	die eine nachvollziehbare und versionierte Entwicklung der Datenbank ermöglichen.
	Die Migration zur Tabelle \Ausgabe{LORA\_sensors} ist im \Anhang{app:Migration} aufgeführt.

	Parallel zur Definition der Tabellen wurden auch die zugehörigen Model-Klassen implementiert,
	welche die Verbindung zwischen Anwendung und Datenbank darstellen.
	Diese Modelle bilden die logische Repräsentation der Datenstrukturen innerhalb der Applikation ab.
	Ein Auszug des Models \Ausgabe{LORASensor} findet sich im \Anhang{app:Model}.


\subsection{Implementierung der Benutzeroberfläche}
\label{sec:ImplementierungBenutzeroberflaeche}
	Die Benutzeroberfläche wurde auf Grundlage der im Abschnitt \ref{sec:Benutzeroberflaeche} dargestellten Mockups entwickelt.
	Für die Umsetzung kam das in Abschnitt \ref{sec:Architekturdesign} vorgestellte Framework NativeScript-Vue zum Einsatz,
	das eine Kombination aus der Vue.js Syntax und nativer App-Entwicklung ermöglicht.

	Vue verwendet eine HTML-ähnliche Syntax, die durch eigene Direktiven wie \Ausgabe{v-for} ergänzt wird,
	um Daten dynamisch darzustellen. Beispielsweise kann mit dieser Direktive einfach über eine Liste von Sensoren iteriert werden:

	\mint{vue}|<li v-for="sensor in sensors"></li>|

	In nativen Anwendungen wie denen, die mit NativeScript erstellt werden, spielt die Performance jedoch eine entscheidende Rolle.
	Da keine klassischen HTML-Elemente verfügbar sind und eine hohe Effizienz beim Rendering notwendig ist,
	kann obiges Beispiel nicht direkt verwendet werden. Stattdessen wird die Komponente \Ausgabe{CollectionView} eingesetzt.
	Diese bietet den Vorteil, dass nur die tatsächlich sichtbaren Listenelemente gerendert werden,
	während Elemente außerhalb des sichtbaren Bereichs verworfen werden.
	Dadurch wird der Speicherverbrauch reduziert und die Anwendung bleibt auch bei großen Datenmengen performant:

	\mint{vue}|<CollectionView :items="dataItems"></CollectionView>|

	Für das Design und Styling der Benutzeroberfläche wird CSS verwendet.
	Einen Eindruck des finalen Designs geben die im \Anhang{app:Screenshots} enthaltenen Screenshots,
	die den Zustand der App nach Abschluss der Implementierungsphase dokumentieren.

	Ein Auszug aus dem Quelltext für die Liste aller Sensoren befindet sich im \Anhang{app:View}.


\subsection{Implementierung der Geschäftslogik}
\label{sec:ImplementierungGeschaeftslogik}
	Die Geschäftslogik bildet das zentrale Bindeglied zwischen den fachlichen Anforderungen und der technischen Umsetzung.
	Sie steuert die Verarbeitung der Anwendungsdaten und definiert die Abläufe zur Umsetzung der funktionalen Anforderungen.
	Im Rahmen dieses Projekts wurden im ersten Schritt sämtliche in
	Abschnitt~\ref{sec:Geschaeftslogik} beschriebenen Komponenten des Laravel-Servers realisiert.

	Für jede relevante Model-Klasse wurde ein zugehöriger Controller erstellt,
	welcher die standardmäßigen Methoden zur Datenverarbeitung gemäß dem \ac{CRUD}-Paradigma bereitstellt.
	In Laravel entspricht dies beispielsweise der Methode \texttt{store} für das Erstellen neuer Einträge (Create).

	Sobald eine Interaktion mit ChirpStack erforderlich war, wurde die jeweilige Funktionalität gezielt im gRPC-Proxy umgesetzt.
	Ziel war es, den Proxy schlank zu halten und ausschließlich um notwendige Funktionen zu erweitern.
	Ein entsprechender Auszug der Proxy-Implementierung ist im \Anhang{app:Proxy} dokumentiert.

	Zur Kommunikation zwischen Laravel und dem gRPC-Proxy wurde eine dedizierte Service-Klasse entwickelt.
	Diese übernimmt sowohl die Authentifizierung als auch den Versand der HTTP-Anfragen an den Proxy.
	Dadurch wird eine klare Trennung der Verantwortlichkeiten erreicht und die Anbindung an externe Dienste zentralisiert.

	Ein Controller-Auszug ist im \Anhang{app:Controller} dargestellt.
	Dieser veranschaulicht unter anderem das Anzeigen und Anlegen eines Geräts.
	Im Rahmen des Anlegens wird dabei auch eine Anfrage an den gRPC-Proxy gesendet,
	um das entsprechende Gerät innerhalb von ChirpStack zu registrieren.


