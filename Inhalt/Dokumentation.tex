% !TEX root = ../Projektdokumentation.tex
\section{Dokumentation}
\label{sec:Dokumentation}
	Die Dokumentation der Anwendung gliedert sich in die Projektdokumentation sowie ein Benutzerhandbuch.
	Die Projektdokumentation beschreibt die im Verlauf des Projekts durchlaufenen Phasen und getroffenen Entscheidungen.

	Das Benutzerhandbuch richtet sich an Endanwender und soll die Bedienung der Anwendung erleichtern.
	Es enthält eine Übersicht über die wichtigsten Funktionen sowie eine schrittweise Anleitung zur Navigation innerhalb der Anwendung.
	Unterstützt durch Screenshots wird die Nutzung anschaulich erklärt.
	Ein Auszug ist im \Anhang{app:BenutzerDoku} zu finden.

	Auf eine separate Entwicklerdokumentation wurde bewusst verzichtet.
	Stattdessen wurde der Quellcode umfangreich kommentiert, sodass sich Entwickler bei einer späteren Weiterentwicklung
	oder Anpassung direkt im Code orientieren können. In der Entwurfsphase wurde ein Komponentendiagramm erstellt,
	das die Struktur und die Abhängigkeiten der wichtigsten Systemkomponenten visualisiert.
	Dieses dient als Überblick für neue Entwickler und ist im \Anhang{app:KomponentenSystem} enthalten.
