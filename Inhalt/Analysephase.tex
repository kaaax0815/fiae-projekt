% !TEX root = ../Projektdokumentation.tex
\section{Analysephase} 
\label{sec:Analysephase}


\subsection{Ist-Analyse} 
\label{sec:IstAnalyse}
	Die Firma \OFF verfolgt das Ziel, eine ganzheitliche \ac{IoT}-Plattform zu entwickeln.
	Im Rahmen dieser Ist-Analyse wird untersucht, in welchem Umfang bereits bestehende Systemkomponenten verfügbar sind
	und wo noch Entwicklungsbedarf besteht.

	Aktuell bietet \OFF ein bestehendes Produkt über eine modulare App an.
	Diese App ist so konzipiert, dass sie flexibel um neue Funktionen erweitert werden kann,
	ohne dass eine vollständige Neuentwicklung erforderlich ist.
	Dies bildet eine solide Grundlage für die geplante Erweiterung zur IoT-Plattform.

	Ein zentrales Element, das derzeit noch fehlt, ist jedoch ein System zur Erfassung und Verarbeitung von Sensordaten.
	Dieses bildet eine essenzielle Voraussetzung für die Realisierung der IoT-Funktionalitäten
	und muss daher im weiteren Projektverlauf neu entwickelt werden.


\subsection{Wirtschaftlichkeitsanalyse}
\label{sec:Wirtschaftlichkeitsanalyse}


	\subsubsection{\gqq{Make or Buy}-Entscheidung}
	\label{sec:MakeOrBuyEntscheidung}
		Die geplante IoT-Plattform soll zentrale Funktionen für die Datenerfassung und -verarbeitung übernehmen,
		insbesondere im Zusammenhang mit Sensordaten, die für zukünftige digitale Dienstleistungen eine strategische Rolle spielen.
		Diese Daten bilden eine wesentliche Grundlage für produktbezogene Auswertungen, Prozessautomatisierung und potenzielle Erlösmodelle.

		Zudem ist eine enge Integration mit der bereits bestehenden modularen App notwendig,
		die unternehmensspezifisch entwickelt wurde. Die App-Architektur erfordert eine hohe Kompatibilität mit bestehenden Schnittstellen
		sowie eine flexible Erweiterbarkeit, um zukünftige Funktionalitäten effizient einbinden zu können.
		Eine externe Lösung würde potenzielle Risiken im Hinblick auf Datenschutz,
		Anpassungsfähigkeit und Abhängigkeit von Drittanbietern mit sich bringen.

		Vor diesem Hintergrund wurde entschieden,
		die erforderlichen Systemkomponenten zur Sensordatenerfassung und -verarbeitung unternehmensintern zu entwickeln.
		Die Eigenentwicklung ermöglicht maximale Kontrolle über sensible Daten,
		volle technische Integration in bestehende Systeme sowie eine langfristig strategische Unabhängigkeit.


	\subsubsection{Projektkosten}
	\label{sec:Projektkosten}
		Im Folgenden werden die voraussichtlichen Kosten des Projekts kalkuliert.
		Dabei sind sowohl die Personalkosten für den zuständigen Entwickler sowie weitere projektbeteiligte Mitarbeitende zu berücksichtigen,
		als auch die Aufwendungen für die unter \Verweis{sec:Ressourcenplanung} aufgeführten Sachmittel und technischen Ressourcen.
		Alle relevanten Kostenpositionen fließen in die Gesamtkalkulation ein,
		um eine realistische Einschätzung des finanziellen Projektaufwands zu ermöglichen.
		
		Die exakten Personalkosten können aus datenschutzrechtlichen Gründen nicht offengelegt werden.
		Daher erfolgt die Kalkulation auf Basis realitätsnaher Durchschnittswerte.
		Für einen voll ausgebildeten Mitarbeitenden werden die Gesamtkosten für das Unternehmen mit \eur{60,00} pro Stunde angesetzt.
		Die Kosten für einen Auszubildenden werden auf Grundlage branchenüblicher Vergütungen
		und betrieblicher Aufwendungen mit \eur{12,00} pro Stunde kalkuliert.
		Zusätzlich wird für die Nutzung von Hard- und Software ein pauschaler Stundensatz von \eur{15,00} berücksichtigt.
		
		Für das Projekt ist ein Gesamtaufwand von 80 Arbeitsstunden eingeplant.
		Auf dieser Grundlage lassen sich die voraussichtlichen Gesamtkosten näherungsweise ermitteln.

		Eine Aufstellung der Kosten befindet sich in Tabelle~\ref{tab:Kostenaufstellung} und sie betragen insgesamt \eur{2739,20}.
		\tabelle{Kostenaufstellung}{tab:Kostenaufstellung}{Kostenaufstellung.tex}


	\subsubsection{Amortisationsdauer}
	\label{sec:Amortisationsdauer}
		Da es sich bei der \ac{IoT}-Plattform um ein neu entwickeltes Produkt handelt,
		kann die Amortisationsdauer nicht auf Basis von Einsparungen durch die Ablösung eines bestehenden Systems berechnet werden.
		Stattdessen orientiert sich die Kalkulation an den erwarteten Einnahmen aus dem geplanten monatlichen Abomodell.

		Die einmaligen Projektkosten belaufen sich auf insgesamt \eur{2.320}.
		Hinzu kommen laufende Betriebskosten von \eur{15} pro Monat (\eur{10} für Serverbetrieb und \eur{5} für Datensicherung)
		Diese Kosten fallen pauschal für das Gesamtsystem an – unabhängig von der Anzahl der Nutzer.

		Bei einem monatlichen Verkaufspreis von \eur{20} pro Kunde ergeben sich folgende Szenarien:

		\begin{itemize}
		\item Bei 1 Kunde: Deckungsbeitrag = \eur{20} - \eur{15} = \eur{5} pro Monat
		\item Bei 5 Kunden: Deckungsbeitrag = $5 \times \eur{20} - \eur{15} = \eur{85}$ pro Monat
		\item Bei 10 Kunden: Deckungsbeitrag = $10 \times \eur{20} - \eur{15} = \eur{185}$ pro Monat
		\end{itemize}

		Die Amortisationsdauer bei 10 Kunden berechnet sich wie folgt:

		\begin{eqnarray}
		\frac{\eur{2.320}}{\eur{185}} \approx 12,54 \ \mbox{Monate}
		\end{eqnarray}

		Somit wäre die Investition bei zehn gleichzeitig zahlenden Kunden bereits nach etwas über einem Jahr vollständig amortisiert.
		Mit wachsender Kundenzahl verkürzt sich die Amortisationsdauer entsprechend weiter.

		Die Analyse zeigt, dass das Projekt bei realistischer Kundengewinnung wirtschaftlich ist.
		Die Investitionskosten können innerhalb eines angemessenen Zeitraums amortisiert werden,
		weshalb die Umsetzung aus wirtschaftlicher Sicht empfohlen wird.


\subsection{Anwendungsfälle}
\label{sec:Anwendungsfaelle}
	Um einen ersten Überblick darüber zu gewinnen, wie Kunden und Mitarbeitende mit der Anwendung interagieren sollen
	und welche Anwendungsfälle aus Sicht der Endnutzer abgedeckt werden müssen, wurde im Rahmen der Analysephase ein Use-Case-Diagramm erstellt.
	Dieses ist im \Anhang{app:UseCase} dargestellt.


\subsection{Lastenheft}
\label{sec:Lastenheft}
	Zum Abschluss der Analysephase wurde gemeinsam mit dem Kunden ein Lastenheft erstellt.
	Darin sind sämtliche Anforderungen des Auftraggebers an die zu entwickelnde Anwendung festgehalten.
	Ein Auszug daraus befindet sich im \Anhang{app:Lastenheft}.
