% !TEX root = ../Projektdokumentation.tex
\section{Projektplanung} 
\label{sec:Projektplanung}


\subsection{Projektphasen}
\label{sec:Projektphasen}
	Für die Umsetzung des Projekts standen insgesamt 80 Stunden zur Verfügung.
	Eine grobe Zeitplanung mit den Hauptphasen lässt sich aus Tabelle~\ref{tab:Zeitplanung} entnehmen.
	Eine detailliertere Zeitplanung findet sich im \Anhang{app:Zeitplanung}.
	
	\tabelle{Zeitplanung}{tab:Zeitplanung}{ZeitplanungKurz}


\subsection{Abweichungen vom Projektantrag}
\label{sec:AbweichungenProjektantrag}

	Im Projektantrag war vorgesehen, eine Weboberfläche zur einfachen Verwaltung der Gateways und Sensoren zu entwickeln.
	Aufgrund der begrenzten Zeit konnte diese Umsetzung jedoch nicht realisiert werden.
	Stattdessen wurde eine Postman Kollektion angelegt um die \ac{API} aufrufen zu können.


\subsection{Ressourcenplanung}
\label{sec:Ressourcenplanung}
	Anschließend wurden alle Ressourcen im \Anhang{app:Ressourcen} aufgelistet. Die Planung umfasst dabei neben allen Hard- und Softwareressourcen,
	die im Rahmen des Projektes verwendet wurden, auch das Personal. Da sich bei \OFF um ein kleines Unternehmen handelt, wurde darauf geachtet,
	dass möglichst wenig Kosten anfallen. Aus diesem Grund wurde das Projekt größtenteils mit kostenloser Software entwickelt.
	Bei kostenpflichtigen Services wurde darauf geachtet, nur die jeweils passende und notwendige Leistung zu erwerben.

\subsection{Entwicklungsprozess}
\label{sec:Entwicklungsprozess}
	Vor Beginn der Umsetzung wählte der Autor einen geeigneten Entwicklungsprozess.
	Die Projektentwicklung orientiert sich grundsätzlich am Wasserfall-Modell,
	da die Anforderungen von \OFF weitgehend klar definiert sind und eine strukturierte, schrittweise Umsetzung ermöglichen.
	Dieses Modell bietet klare Phasen und erleichtert die Planung sowie Nachvollziehbarkeit des Projektfortschritts\footnote{\Vgl \citet{Wasserfallmodell}}.
