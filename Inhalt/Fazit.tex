% !TEX root = ../Projektdokumentation.tex
\section{Fazit} 
\label{sec:Fazit}

\subsection{Soll-/Ist-Vergleich}
\label{sec:SollIstVergleich}
	Durch eine sorgfältige Anforderungsanalyse und eine strukturierte Entwurfsphase konnte das Projekt vollständig nach Plan umgesetzt werden.
	Zu Beginn wurden vergleichbare Anwendungen und Systeme umfassend recherchiert,
	um bewährte Konzepte und Technologien in die Entwicklung einfließen zu lassen.
	Diese Vorarbeit bildete die Grundlage für eine effiziente und zielgerichtete Umsetzung.

	Wie Tabelle~\ref{tab:Vergleich} zeigt, konnten alle Phasen des Projekts ohne Abweichungen im vorgesehenen Zeitraum abgeschlossen werden.
	
	\tabelle{Soll-/Ist-Vergleich}{tab:Vergleich}{Zeitnachher}

\subsection{Lessons Learned}
\label{sec:LessonsLearned}
	Im Verlauf des Projekts konnte der Prüfling durch eigenständige Recherchen sowie die Entwicklung eigener Lösungsansätze
	wertvolle praktische Erfahrungen sammeln. Eine zentrale Erkenntnis war die Bedeutung einer fundierten Planung und eines durchdachten Entwurfs.
	Diese bilden die Grundlage für eine strukturierte und effiziente Umsetzung,
	wodurch Projekte reibungsloser realisiert und Ressourcen gezielt eingesetzt werden können.

	Positiv zu bewerten ist außerdem die Fähigkeit, auftretende Probleme eigenständig zu analysieren und passende Lösungen zu entwickeln.
	Dies stellt eine wichtige Kompetenz für die erfolgreiche Durchführung zukünftiger Projekte dar.


\subsection{Ausblick}
\label{sec:Ausblick}
	Da die Anwendung modular und erweiterbar konzipiert wurde, sind zukünftige Erweiterungen problemlos möglich.
	Erste Ideen für zusätzliche Funktionen bestehen bereits, beispielsweise die Entwicklung einer Weboberfläche,
	die das Anlegen und Verwalten von Sensordaten erleichtert. Auch die Möglichkeit,
	Sensordaten mithilfe benutzerdefinierter Formeln direkt in der Anwendung zu modifizieren, ist angedacht.

	Darüber hinaus bestehen Potenziale für eine technische Weiterentwicklung,
	etwa durch die Integration weiterer Übertragungstechnologien.
	Ein mögliches Szenario ist die Einbindung von NB-IoT, einem auf Mobilfunk basierenden IoT-Netzwerk,
	das im Gegensatz zu LoRaWAN keine eigene Gateway-Infrastruktur erfordert.
	Dadurch ließen sich neue Anwendungsbereiche erschließen und die Flexibilität der Lösung weiter erhöhen.
